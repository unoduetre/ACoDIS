\documentclass{classrep}
\usepackage{polski}
\usepackage{ucs}
\usepackage[utf8x]{inputenc}
\usepackage[T1]{fontenc}
\usepackage{cite}
%\usepackage{tcolorbox}
%\usepackage{apacite}
\usepackage{cancel}
\usepackage{graphicx}
\usepackage{dcolumn}
\usepackage{color}
\usepackage{colortbl}
\usepackage{enumerate}
\usepackage{url}
\usepackage[squaren]{SIunits}
\usepackage{icomma}
\usepackage{hyperref}
%\usepackage{url}
\usepackage{float}
\usepackage{indentfirst}
\usepackage{amssymb}
\usepackage{verbatim}
\usepackage{tabulary}
\usepackage{longtable}
\usepackage{rotating}
\usepackage{formular}
\usepackage{marginnote}
\usepackage{listingsutf8}
\usepackage{ifpdf}
%\usepackage{natbib}
\usepackage{fancybox}
\lstset{inputencoding=utf8/latin2,breaklines=true}
\newFRMfield{wyraz}{2cm}
\newif\ifShowAnswers
%\ShowAnswerstrue
\newcommand{\fillin}[2]{\ifShowAnswers{#1}\else{\marginnote{#2}\useFRMfield{wyraz}}\fi}
\newtheorem{xdefinicja}{Definicja}
\newenvironment{definicja}{\begin{xdefinicja}\normalfont}{\end{xdefinicja}}
\ifpdf
\newcommand{\obraz}[3]{
\begin{figure}[H]
\centering
\includegraphics[width=9cm]{#1.pdf}
\caption{#3}
\label{fig:#2}
\end{figure}
}
\newcommand{\malyobraz}[3]{
\begin{figure}[H]
\centering
\includegraphics[width=4cm]{#1.pdf}
\caption{#3}
\label{fig:#2}
\end{figure}
}
\newcommand{\podwojnyobraz}[4]{
\begin{figure}[H]
\centering
\includegraphics[width=4cm]{#1.pdf}
\includegraphics[width=4cm]{#2.pdf}
\caption{#4}
\label{fig:#3}
\end{figure}
}
\newcommand{\poziomyobraz}[2]{
\begin{sidewaysfigure}
\centering
\includegraphics[width=25cm]{#1.pdf}
%\caption{#2}
\label{fig:#2}
\end{sidewaysfigure}
}
\else
\newcommand{\obraz}[3]{
\begin{figure}[H]
\centering
\includegraphics[width=9cm]{#1.eps}
\caption{#3}
\label{fig:#2}
\end{figure}
}
\newcommand{\malyobraz}[3]{
\begin{figure}[H]
\centering
\includegraphics[width=4cm]{#1.eps}
\caption{#3}
\label{fig:#2}
\end{figure}
}
\newcommand{\podwojnyobraz}[4]{
\begin{figure}[H]
\centering
\includegraphics[width=4cm]{#1.eps}
\includegraphics[width=4cm]{#2.eps}
\caption{#4}
\label{fig:#3}
\end{figure}
}
\newcommand{\poziomyobraz}[2]{
\begin{sidewaysfigure}
\centering
\includegraphics[width=25cm]{#1.eps}
%\caption{#2}
\label{fig:#2}
\end{sidewaysfigure}
}
\fi
\newcommand{\rysunek}[4]{
\begin{figure}[H]
\centering
\begin{picture}#3
#4
\end{picture}
\caption{#2}
\label{fig:#1}
\end{figure}
}
\newcommand{\tabela}[4]{
\begin{table}[H]
\centering
\begin{tabular}[t]{#3}
#4
\end{tabular}
\caption{#2}
\label{tab:#1}
\end{table}
}

\newcommand{\tabelap}[4]{
\begin{table}[p]
\centering
\begin{tabular}[t]{#3}
#4
\end{tabular}
\caption{#2}
\label{tab:#1}
\end{table}
}

\newcommand{\tabelah}[4]{
\begin{table}[ht]
\centering
\begin{tabular}[t]{#3}
#4
\end{tabular}
\caption{#2}
\label{tab:#1}
\end{table}
}

\newcommand{\tabelat}[4]{
\begin{table}[t]
\centering
\begin{tabular}[t]{#3}
#4
\end{tabular}
\caption{#2}
\label{tab:#1}
\end{table}
}

\newcommand{\tabelab}[4]{
\begin{table}[b]
\centering
\begin{tabular}[t]{#3}
#4
\end{tabular}
\caption{#2}
\label{tab:#1}
\end{table}
}

\newcommand{\ltabela}[3]{
\clearpage
\begin{longtable}[c]{#2}
\caption{#1}\\
#3
\end{longtable}
}

\lstdefinelanguage{Smalltalk}{
  morekeywords={true,false,self,super,nil},
  sensitive=true,
  morecomment=[s]{"}{"},
  morestring=[d]',
}
\lstdefinestyle{SmalltalkStyle}{
  literate={:=}{{$\gets$}}1{^}{{$\uparrow$}}1
} 

\usepackage{amsfonts}
\usepackage[intlimits]{amsmath}


\studycycle{Informatyka, studia dzienne}
\coursesemester{II}

\coursename{Pracownia Problemowa}
\courseyear{2011/2012}

\courseteacher{Prowadzący}
\coursegroup{dzień, godzina}

\author{
  \studentinfo{Autor 1}{nr albumu 1} \and
  \studentinfo{Autor 2}{nr albumu 2} \and
  \studentinfo{Autor 3}{nr albumu 3} \and
  \studentinfo{Autor 4}{nr albumu 4} \and
  \studentinfo{Autor 5}{nr albumu 5}
}

\title{Temat projektu}


\begin{document}
\maketitle

\section{Problem}

W tej sekcji należy umieścić krótki opis problemu, który jest przedmiotem realizowanego projektu (2-4 zdania).


\section{Wprowadzenie}

W ramach tej sekcji należy przedstawić problematykę związaną z realizowanym projektem. Można opisać punkt wyjścia, okoliczności powstania problemu, niewystarczalność istniejących rozwiązań.


\section{Aktualny stan wiedzy}

Należy tu przedstawić aktualny stan wiedzy na tematy związane z rozwiązywanym problemem. Należy zaprezentować teorię potrzebną do rozwiązania problemu (wraz z odwołaniami do pozycji bibliograficzych, z których uzyskano wiedzę na dany temat, a które zostaną umieszczone w ostaniej sekcji "Bibliografia"). Zamieszczony opis teorii ma być opisem własnym, a nie skopiowanym.

Można opisać istniejące rozwiązania w ramach danej tematyki i ewentualnie przedstawić ich zalety i wady (jeżeli istnieją).


\section{Opis rozwiązania problemu}

W tej sekcji należy przedstawić zaproponowane rozwiązanie postawionego problemu:
\begin{itemize}
\item koncepcje, metody, podejścia użyte do stworzenia rozwiązania,
\item własną koncepcję rozwiązania (popartą podlądowymi rysunkami),
\item architekturę rozwiązania,
\item zależności poszczególnych komponentów rozwiązania,
\item przykładowy proces działania zaproponowanego rozwiązania w ramach opisu teoretyczno-praktycznego,
\item krótki opis ewaluacji stworzonego rozwiązania, czyli przykładowo, aby udowodnić poprawność zrealizowanego rozwiązania problemu, stworzono przykładowy system lub przykładową aplikację o takiej a takiej funkcjonalności,
\item ...
\end{itemize}

Przedstwiony opis rozwiązania powinien być poparty licznymi rysunkami.


\section{Opis implementacji}

Należy tu zamieścić krótki opis zaprojektowanych elementów (komponentów) składających się na implementację zaproponowanego rozwiązania. Do tego celu można wykorzystać diagramy UML. Należy także podać technologie, które użyto do budowy systemu lub aplikacji.

Następnie należy przestawić opis zrealizowanego systemu lub aplikacji w postaci dokumentacji użytkownika.


\section{Dyskusja rozwiązania i podsumowanie}

Sekcja ta powinna zawierąć dyskusję stworzonego rozwiązania w odniesieniu do istniejących koncepcji, podejść, metod oraz technik i technologii. Każdy wniosek powinien mieć poparcie we wcześniej przeprowadzonych badaniach (zarówno teoretycznych, jak i praktycznych).

Należy również omówić i wyjaśnić wszystkie napotkane problemy, jeżeli takie były.

Następnie należy podsumować powyższe wnioski oraz zaproponować ewentualne kierunki rozwoju stworzonego rozwiązania.

\section{Udział w projekcie}

Proszę przedstwić udział poszczególnych osób w realizacji projektu: 
\begin{itemize}
\item podział prac,
\item zadania, czynności zrealizwoane przez każdą osobę.
\end{itemize}


\begin{thebibliography}{10}
\end{thebibliography}
Należy podać wykorzystaną w ramach projektu literaturę. Każda pozycja bibliograficzna powinna być wykorzystana zgodnie ze stanem faktycznym w powyższej dokumentacji.


\end{document}
